 \documentclass[a4paper,12pt]{article}

\usepackage{graphicx}          % permite incluir eps no documento
\usepackage{latexsym}
\usepackage{amsmath}
\usepackage{amssymb}           % carrega letras matemáticas
\usepackage{mathrsfs}          % carrega letras matemáticas
\usepackage{hyperref}
\usepackage{verbatim}
\usepackage{xcolor,listings}

%%%%%%%%%%%%%%%%%%%%%%%%%%%%%%%%%%%%%%%%%%%%%%%%%%%%%%%%%%%%%%%%
% commands /macros

\newcommand{\GS}{\mathbb{GS}}

\newcommand{\bash}[1]{\small\textbf{\lstinline§> #1§}\\} 

\newcommand{\out}[1]{\small\lstinline§ #1§}

\newcommand{\hsklcmt}[1]{~~-- \footnotesize\textit{#1}} 

% use Courier font (pcr)
\newcommand{\haskellCode}{\fontfamily{pcr}\selectfont} 

\newenvironment{sgcode}
	{ \haskellCode
	  \begin{tabular}{|p{0.9\textwidth}|}
      \hline	
	}
	{ \\\hline  
      \end{tabular} \\
	  \par 
	}
 
%%%%%%%%%%%%%%%%%%%%%%%%%%%%%%%%%%%%%%%%%%%%%%%%%%%%%%%%%%%%%%%%

\begin{document}

\title{Scoring Games Calculator}
\author{Jo\~{a}o Pedro Neto, Carlos Santos,\\ Urban Larson, Richard Nowakowski}
\date{October 9, 2015 draft}
\maketitle

\newpage

\tableofcontents

%%%%%%%%%%%%%%%%%%%%%%%%%%%%%%%%%%%%%%%%%%%%%%%%%%%%%%%%%%%%%%%%
%%%%%%%%%%%%%%%%%%%%%%%%%%%%%%%%%%%%%%%%%%%%%%%%%%%%%%%%%%%%%%%%
%%%%%%%%%%%%%%%%%%%%%%%%%%%%%%%%%%%%%%%%%%%%%%%%%%%%%%%%%%%%%%%%
\newpage \section*{Introduction}

Haskell is a functional language with lazy evaluation and static typing.

The Scoring Game Calculator (SGC) is implemented in Haskell, and it's run on 
the Haskell interpreter, provided the adequate module is imported. 

This means the user can mix Haskell code and SGC functions to perform computations.

\textcolor{red}{Warning: The program and this doc are in version 0.03 beta. As usual, bugs and typos are expected.}

%%%%%%%%%%%%%%%%%%%%%%%%%%%%%%%%%%%%%%%%%%%%%%%%%%%%%%%%%%%%%%%%
%%%%%%%%%%%%%%%%%%%%%%%%%%%%%%%%%%%%%%%%%%%%%%%%%%%%%%%%%%%%%%%%
%%%%%%%%%%%%%%%%%%%%%%%%%%%%%%%%%%%%%%%%%%%%%%%%%%%%%%%%%%%%%%%%
\section{Haskell Nanotutorial}

This section provides a very short tutorial on Haskell, so that the user can
take advantage of the language potential.

The basic Haskell data structure is the list. Lists are sequences of
same type data. The next are egs of lists of integers:

\begin{sgcode}
\bash{[1..5]}
\out{[1,2,3,4,5]} \\
\bash{[1,2,5] ++ [3,4]} 
\out{[1,2,5,3,4]} \hsklcmt{operator ++ concatenates lists}
\end{sgcode}

Other types are strings like "abc" (which are just lists of characters) 
and tuples like (1,2) or ('a',1,1.0) that are able to contain values of different types.

There's a special construct to create more complex lists, similar to set definitions in Math, called list comprehensions:

\begin{sgcode}
\bash{[x^2 | x <- [1..5]]}
\out{[1,4,9,16,25]} \\
\bash{[(x,y) | x <- [1,2], y <- "abc"]} 
\out{[(1,'a'),(1,'b'),(1,'c'),(2,'a'),(2,'b'),(2,'c')]}
\end{sgcode}

The main operation in Haskell is function application, like \verb|f x| (no parenthesis):

\begin{sgcode}
\bash{length [1,-2,3]}
\out{3} \hsklcmt{-- length returns list size} \\
\bash{reverse [1,-2,3]}
\out{[3,-2,1]} \\
\bash{[10,20,30]!!1}
\out{20}\hsklcmt{get (i+1)-th element, same as (!!)[10,20,30] 1}
\end{sgcode}

Haskell has lambda-expressions:

\begin{sgcode}
\bash{let positive = \\x -> x>0}
\bash{positive 3}
\out{True} \\
\bash{let succ = \\x -> x+1}
\bash{succ 3}
\out{4} \\
\bash{let add3 = \\x y z -> x+y+z}
\bash{add3 1 2 3}
\out{6} \\
\bash{let make = \\n -> LE n [Nu (n+1)]}
\bash{make 0}  
\out{<^0|1>} \hsklcmt{see next section about Nu \& LE}
\end{sgcode}

It's also possible to define new operators:

\begin{sgcode}
\bash{let (-->) a b = not a \|\| b    -- define implies}
\bash{False --> True}
\out{True}
\end{sgcode}

More on lambda-expressions: \url{http://www.cs.bham.ac.uk/~vxs/teaching/Haskell/handouts/lambda.pdf}

Functions can be high-order, ie, they can receive functions as arguments:

\begin{sgcode}
\bash{filter positive [-1,2,-3,4]}
\out{[2,4]}  \\
\bash{map (\\x -> 2*x) [10,20,30]}
\out{[20,40,60]} \\
\bash{map make [1,2,3]}
\out{[<^1|2>,<^2|3>,<^3|4>]}
\end{sgcode}

Also, we can apply composition of functions using operator '.':

\begin{sgcode}
\bash{map ((\\x -> 2*x).succ) [10,20,30]}
\out{[22,42,62]}
\end{sgcode}

The operator \verb|$| is the same has function application but with lower
precedence. It is useful to save the writing of parenthesis:

\begin{sgcode}
\bash{length (map (\\x -> 2*x) (filter positive [1,2,3]))}
\out{3}\\
\bash{length \$ map (\\x -> 2*x) \$ filter positive [1,2,3]}
\out{3}
\end{sgcode}

%%%%%%%%%%%%%%%%%%%%%%%%%%%%%%%%%%%%%%%%%%%%%%%%%%%%%%%%%%%%%%%%
%%%%%%%%%%%%%%%%%%%%%%%%%%%%%%%%%%%%%%%%%%%%%%%%%%%%%%%%%%%%%%%%
%%%%%%%%%%%%%%%%%%%%%%%%%%%%%%%%%%%%%%%%%%%%%%%%%%%%%%%%%%%%%%%%
\section{Instructions}


%%%%%%%%%%%%%%%%%%%%%%%%%%%%%%%%%%%%%%%%%%%%%%%%%%%%%%%%%%%%%%%%
\subsection{Scoring Module}

This module includes the apparatus to create and manipulate scoring games.

To access this module please load module \verb|Scoring.hs|. 

To list all available functions insert:

\begin{sgcode}
\bash{commands}
\hsklcmt{big list of functions}
\end{sgcode}

To get individual help about one of those functions, say rightOp:

\begin{sgcode}
\bash{help "rightOp"}
\hsklcmt{function description}
\end{sgcode}

What follows is a description of the module's available functions.

\subsubsection{Constructors}

There are four constructors of scoring games:

\begin{itemize} \setlength\itemsep{0.1em}
  \item \verb|Nu n| endgame $\equiv \{ \emptyset^n | \emptyset^n \}$
  \item \verb|BE n m| endgame $\equiv \{ \emptyset^n | \emptyset^m \}, n\leq m$
  \item \verb|LE n g| left endgame $\{ \emptyset^n | [\text{list of games}] \}$
  \item \verb|RE g n| right endgame $\{ [\text{list of games}] | \emptyset^n \}$
  \item \verb|Op gL gR| game with left and right options $\{ [\text{list of games}] | [\text{list of games}] \}$
\end{itemize}

Some egs:

\begin{sgcode}
\bash{Nu 4}
\bash{Nu (-2)}
\bash{let g1 = LE 3 [Nu 4]}
\bash{g1}
\out{<^3|4>} \\
\bash{RE [g1] 0}
\out{<<^3|4>|^0>} \\
\bash{Op [Nu 1, g1] [Nu (-1)]}
\out{<1,<^3|4>|-1>}
\end{sgcode}

Notice how lists intermix with game constructors. This is necessary since the left/right options
of a scoring game can contain several different games. They are thus organized in lists of games.

After an assignment, a game is presented in the format \verb!<...|...>!. If the user wishes to 
see the true game representation, function \verb|showRaw| shows it:

\begin{sgcode}
\bash{let g1 = LE (-3.1) [Nu 4.5]}
\bash{g1}
\out{<^-3.10|4.50>} \\
\bash{showRaw g1}
\out{"LE (-3.10) [Nu 4.50]"}
\end{sgcode}

It's possible to represent \verb|Nu n| just by \verb|n|, but only when \verb|n| is an integer:

\begin{sgcode}
\bash{let g1 = LE 3 [4]}
\bash{g1}
\out{<^3|4>} \\
\bash{showRaw g1}
\out{"LE 3 [Nu 4]"}
\end{sgcode}

A faster way to write games is to use the description directly. For this it is necessary
to enclose the game within \verb|g"..."|:

\begin{sgcode}
\bash{let g1 = g"<2|<^-3|-2>,<4|-4>,-5>"}
\bash{g1}
\out{<2|-5,<^-3|-2>,<4|-4>>} \\
\bash{showRaw g1}
\out{"Op [Nu 2] [Nu (-5),LE (-3) [Nu (-2)],Op [Nu 4]...}
\bash{let g1 = g"<-1/2|^-11/128>"}
\bash{g1}
\out{<-0.5|^-0.0859375>}
\end{sgcode}

Notice that the game options might be rearranged by the program. If the user
wishes to see the non-canonical version of the game it should use \verb|gg"..."|
instead. It's not advisable to use \verb|g| or \verb|gg| as variables (the same
warning applies to any function name included in the calculator).

\subsubsection{Game options manipulation}

\begin{itemize} \setlength\itemsep{0.1em}
  \item \verb|leftOp game| returns a list with the left options of a given game
  \item \verb|rightOp game| returns a list with the right options of a given game
\end{itemize}

\begin{sgcode}
\bash{let g1 = g"<1,<^2|3>|4>"}
\bash{g1}
\out{<1,<^2|3>|4>}\\
\bash{leftOp g1}
\out{[1,<^2|3>]} \\
\bash{rightOp g1}
\out{[4]}
\end{sgcode}

There are two other functions useful to extract a specific game from the original game structure.

\begin{itemize} \setlength\itemsep{0.1em}
  \item \verb|lop n game| returns the n-th left option
  \item \verb|rop n game| returns the n-th right option
\end{itemize}

\begin{sgcode}
\bash{g1}
\out{<<1|1,<1|1>>,<2|3>|<<<^4|4>|<4|^4>>|-5>>}\\
\bash{lop 1 \$ g1}
\out{<1|1,<1|1>>} \\
\bash{rop 2 . lop 1 \$ g1}
\out{<1|1>} \\
\bash{rop 1 . lop 2 \$ g1}
\out{3}
\end{sgcode}

The \verb|goto| function travels down the game tree following a list of indexes. In this list, a positive
number means go to the left options, a negative to the right options (zero is not allowed). If the index
does not exist, an error is produced.

\begin{sgcode}
\bash{g1}
\out{<<^0|<-1|-2>>|<<-1|-2>,<-1|^2>|<-2|^1>>>}\\
\bash{goto [1] g1}
\out{<^0|<-1|-2>>} \hsklcmt{same as lop 1 g1} \\
\bash{goto [1,-1] g1}
\out{<-1|-2>} \hsklcmt{same as rop 1 . lop 1 \$ g1} \\
\bash{goto [1,-1,1] g1}
\out{-1} \hsklcmt{same as lop 1 . rop 1 . lop 1 \$ g1}
\end{sgcode}

Functions \verb|addLop| and \verb|addRop| add a new game as the n-th game option:

\begin{sgcode}
\bash{g1}
\out{<<2|2,<2|2>>,<1|1>|<<<^4|4>|<4|^4>>|-5>>}\\
\bash{addRop 1 g1 \$ 7}
\out{<<2|2,<2|2>>,<1|1>|7,<<<^4|4>|<4|^4>>|-5>>} \\
\bash{addLop 2 g1 7}
\out{<<2|2,<2|2>>,7,<1|1>|<<<^4|4>|<4|^4>>|-5>>}
\end{sgcode}

Functions \verb|remLop| and \verb|remRop| delete the n-th game option:

\begin{sgcode}
\bash{g1}
\out{<<2|2,<2|2>>,<1|1>|<<<^4|4>|<4|^4>>|-5>>}\\
\bash{remLop 2 g1}
\out{<<2|2,<2|2>>|<<<^4|4>|<4|^4>>|-5>>} \\
\bash{remRop 1 g1}
\out{*** Exception: Cannot delete only game option}
\end{sgcode}

\subsubsection{stable}

Function \verb|stable| checks if a given game description is stable, ie, $stable(G) \iff ls(G) \leq rs(G)$:

\begin{sgcode}
\bash{stable \$ BE 2 1}
\out{False} \\
\bash{stable \$ RE [Op [Nu 4] [Nu (-3)]] (-2)}
\out{True}
\end{sgcode}


\subsubsection{guaranteed}

Function \verb|guaranteed| checks if a given game $G$ description is guaranteed, ie $G \in \GS$:

\begin{sgcode}
\bash{guaranteed \$ RE [Op [Nu 4] [Nu (-3)]] (-2)}
\out{False} \\
\bash{guaranteed \$ RE [Op [Nu (-4)] [Nu (-3)]] (-2)}
\out{True}
\end{sgcode}

There are also functions \verb|hot|, \verb|tepid| and \verb|zugzwang| that verify if the respective
condition holds for a scoring game.

\subsubsection{rank}

Function \verb|rank| returns the game's birthday:

\begin{sgcode}
\bash{rank 412}
\out{0} \\
\bash{rank \$ g"<<^1|2>,2|<1|1>>"}
\out{2}
\end{sgcode}

\subsubsection{conjugate}

The conjugate of a game $G = \{ G^\mathcal{L} | G^\mathcal{R} \}$ is $-G = \{ -G^\mathcal{R} | -G^\mathcal{L} \}$:

\begin{sgcode}
\bash{let g1 = RE [Nu 4] (-1)}
\bash{conjugate g1}
\out{<^1|-4>} \\
\bash{-g1}
\out{<^1|-4>} \hsklcmt{same thing}
\end{sgcode}

\subsubsection{stops}

\begin{itemize} \setlength\itemsep{0.1em}
  \item \verb|ls game| left stop of game
  \item \verb|rs game| right stop of game
  \item \verb|ls_d game| $\underline{Ls}(game)$
  \item \verb|ls_u game| $\overline{Ls}(game)$
  \item \verb|rs_d game| $\underline{Rs}(game)$
  \item \verb|rs_u game| $\overline{Rs}(game)$
\end{itemize}

\begin{sgcode}
\bash{let g1 = g"<^-5|<<-1|1>|3>>"}
\bash{rs g1}
\out{1.0} \\
\bash{ls g1}
\out{-5.0}
\end{sgcode}

\subsubsection{r-protected}

\begin{itemize}
  \item \verb|lrp game| checks if a game is left-r-protected
  \item \verb|rrp game| checks if a game is right-r-protected
\end{itemize}

\begin{sgcode}
\bash{lrp 2 (Nu 3)}
\out{True}
\end{sgcode}

\subsubsection{relational operators}

There are two sets of relational operators.

The first set compares two games, returning true if the condition is satisfied, or false otherwise.

\begin{itemize} \setlength\itemsep{0.1em}
  \item \verb|>==| 
  \item \verb|<==| 
  \item \verb|===|  
  \item \verb|/==| , the negation of \verb|===| 
\end{itemize}

\begin{sgcode}
\bash{Nu 3 >== Nu 2}
\out{True} \\
\bash{5 >== g"<^-4|6>"}
\out{True} 
\end{sgcode}

The second set compares a game to a number:

\begin{itemize} \setlength\itemsep{0.1em}
  \item \verb|>=.| , means $G \succeq n$
  \item \verb|<=.| , $G \preceq n$
  \item \verb|>.| , $G \succ n$
  \item \verb|<.| , $G \prec n$
  \item \verb|==.| , $G \succeq n \wedge G \preceq n$ 
  \item \verb|/=.| , the negation of \verb|==.|
\end{itemize}

A game $g \succeq n \iff$ \verb|lrp n g| is true.

\begin{sgcode}
\bash{Nu 3 >=. 2}
\out{True} \\
\bash{Nu 3 <=. 2}
\out{False} 
\end{sgcode}

This second set of relational operators is redundant, but uses a faster algorithm to compute the result.

\subsubsection{disjunctive sum of games}

There are three operators available \verb|#|, \verb|+| and \verb|-|.

Operator \verb|#| performs the disjunctive sum, just like \verb|+|, but without simplifications

\begin{sgcode}
\bash{let g1 = LE (-3) [Op [2] [1]]}
\bash{let g2 = RE [-3] 0}
\bash{g1\#g2}
\out{<<^-6|<-1|-2>>|<<-1|^2>,<-1|-2>|<-2|^1>>>} \\
\bash{g1+g2}
\out{<<^-6|^-1>|^2>} \\
\bash{g1-g2}
\out{<^-3|<^0|^5>>}
\end{sgcode}

\subsubsection{invertible}

A game $G$ is invertible if $G-G=0$.

\begin{sgcode}
\bash{let g1 = Op [2] [2]}
\bash{g1-g1}
\out{0} \\
\bash{invertible g1}
\out{True}
\end{sgcode}

\subsubsection{canonize}

When a game is made using the constructors, it might not be in its canonical form. 

Function \verb!canonize! performs that operation:

\begin{sgcode}
\bash{g1}
\out{<<2|2,<2|2>>,<1|1>|<<<^4|4>|<4|^4>>|-5>>} \\
\bash{canonize g1}
\out{<^2|^4>} \hsklcmt{an alternative is to evaluate g1+0}
\end{sgcode}

\subsubsection{Conway games}

The next functions embed Conway games into scoring games.

SGC stands for Short (ie, finite) Conway Game.

\begin{itemize} \setlength\itemsep{0.1em}
  \item \verb|scgInt n| or \verb|hat n| transform a integer Conway's game into a scoring game
  \item \verb|scgDiatic (num,den)| transform a diatic Conway's game into a scoring game
  \item \verb|zeta num den| order preserving embedding for numbers (integers and diatics) in the format \verb|num/den|
  \item \verb|scgStar n| star game \verb|*n| into a scoring game
  \item \verb|up| ie $\{0|*\}$
  \item \verb|down| ie $\{*|0\}$
  \item \verb|star| ie $\{0|0\}$ (there's also \verb|star2| and \verb|star3| available)
\end{itemize}

Notice that a diatic is a fraction like $a/2^b$, egs: 1/4, 5/16, -7/32, 1/1.

\begin{sgcode}
\bash{scgInt (-5)}
\out{<^0|<^0|<^0|<^0|<^0|0>>>>>} \\
\bash{hat (-5)}
\out{<^0|<^0|<^0|<^0|<^0|0>>>>>} \\
\bash{scgDiatic (2,8)}
\out{<<0|<0|<0|<0|^0>>>>|<<0|<0|<0|^0>>>|<0|<0|^0>>>>}
\bash{zeta (-1) 32}
\out{<<<<<<^0|0>|0>|0>|0>|0>|0>} \hsklcmt{scgDiatic (-1,32)} \\
\bash{scgStar 2}
\out{<<0|0>,<<0|0>|<0|0>>|<0|0>,<<0|0>|<0|0>>>} \\
\bash{star2}
\out{<<0|0>,<<0|0>|<0|0>>|<0|0>,<<0|0>|<0|0>>>} \\
\bash{let g1 = Op [star, hat 1][LE 0 [down]]}
\bash{g1}
\out{<<0|0>,<0|^0>|<^0|<<0|0>|0>>>} \\
\bash{canonize g1}
\out{<<0|^0>|<^0|0>>}
\end{sgcode}

\subsubsection{converting to latex}

This command creates a latex description of a game

\begin{sgcode}
\bash{latex \$ g"<-3|^12>"}
\out{"$$<-3|\\emptyset^\{12\}>$$"}
\end{sgcode}

This can be copy-pasted to a latex file and shown as $<-3|\emptyset^{12}>$.

\subsubsection{testing properties}

It is possible to verify if some game properties hold for a large set of
random generated games. The program might be able to find a counter-example
(which would negate the property). If not, while not proving the property,
the user may become more confident and try to demonstrate it mathematically.

There are two testing functions:

\begin{itemize} \setlength\itemsep{0.1em}
  \item \verb|test| for unary propositions, it receives a proposition (using a lambda-expression) 
  and a number determining the number of required tests
  \item \verb|test2| works the same way but for binary propositions
\end{itemize}

Notice that only guaranteed games will be tested.

An eg using \verb!test!:

\begin{sgcode}
\bash{test (\\g -> ls g == ls_d g) 100}
\out{Found a counter example:} \\
\out{Game: <<2|^5>|^5>} \\
\bash{let g1 = g"<<2|^5>|^5>"}
\bash{ls g1}
\out{5} \\
\bash{ls_d g1}
\out{2}
\end{sgcode}

Let's see an eg where the function does not find a counter-example:

\begin{sgcode}
\bash{test (\\g -> ls g >= ls_d g) 10}
\out{..........Tests finished!}
\end{sgcode}

Function \verb|test2| receives a lambda expression with two games:

\begin{sgcode}
\bash{test2 (\\g1 g2 -> g2 <=. ls_d (g1+g1)) 20}
\out{Found a counter example:} \\
\out{Game 1: -3} \\
\out{Game 2: -2} 
\end{sgcode}

Another example: say we have the following conjecture (it is a theorem, but let's suppose) saying:

If $G_1, G_2$ are guaranteed scoring games then,

$$\underline{Ls}(G_1+G_2) \leq \underline{LS}(G_1) + \underline{Ls}(G_2)$$

we can try to find a counter-example like this:

\begin{sgcode}
\bash{let prop g1 g2 = ls_d (g1+g2) <= ls_d g1 + ls_d g2}
\bash{test2 prop 35}
\out{.....................................Tests finished!} 
\end{sgcode}

It's possible that the generator produces large games, which might take
a while to sum or canonize. On those cases, the user might interrupt the interpreter
using the \verb|CTRL+C| key combination.

As said, \verb|test| and \verb|test2| generate guaranteed and canonized games.

~

Functions \verb|testn| and \verb|test2n| work the same way but include a random number, which
is useful to test propositions using a number.

\begin{sgcode}
\bash{let prop g n = ls_d(g+LE n [Nu\$n+1]) == ls_d g+n}
\bash{testn prop 30}
\out{................................Tests finished!} 
\end{sgcode}

It's possible to return one guaranteed and canonized random game:

\begin{sgcode}
\bash{getCanonize}
\out{<<4|-3>|<^-5|-2>,<^-4|^3>>} \\
\bash{getCanonize}
\out{<4|^6>} \\
\bash{liftM showRaw getCanonize}
\out{"RE [Nu 4] 8"} \\
\bash{liftM showRaw getCanonize}
\out{"Op [Nu 0,BE (-9) 7] [BE (-1) 5]"}
\end{sgcode}

Or even to return a list of them:

\begin{sgcode}
\bash{getCanonizeList 4}
\out{[5,<^-10|^-9>,<1|-3>,0]} \hsklcmt{list of four games} \\ 
\bash{liftM (map showRaw) \$ getCanonizeList 4}
\out{["BE 4 10","Nu 3","Nu 0","Nu 8"]}
\end{sgcode}

%It's possible to access lists of random guaranteed and canonized games using \verb|canonizedSample|
%which receives a random seed (an integer value that allows repeatability) and a number
%of games:
%
%\begin{sgcode}
%\bash{canonizedSample 101 4}
%\out{[<^-44|^58>,<64|^94>,-56,95]} 
%\end{sgcode}
%
%There's also \verb|guaranteedSample| which outputs only guaranteed games (before canonization) and 
%\verb|unguaranteedSample| which are random games with any type of structure.
%
%The next code checks if the produced games verify some known properties:
%
%\begin{sgcode}
%\bash{map guaranteed \$ unguaranteedSample 1 10}
%\out{[False,False,True,False,False,True,True,False]} \\
%\bash{map hot \$ unguaranteedSample 1 10}
%\out{[True,True,False,True,True,False,False,True]}  \\
%\bash{map stable \$ filter (not.isOp) \$ unguaranteedSample 1 10}
%\out{[False,False,True,False,True,True]}
%\end{sgcode}

\subsubsection{converting strings into games}

To translate a raw description string into a game:

\begin{sgcode}
\bash{let description = "BE (-3) 7"}
\bash{description}
\out{"BE (-3) 7"} \\
\bash{(read description)::Game}
\out{<^-3|^7>} \\
\bash{let gs = ["BE (-3) 7","Nu (-6)","LE 5 [Nu 8]"]}
\bash{(map read gs)::[Game]}
\out{[<^-3|^7>,-6,<^5|8>]} 
\end{sgcode}

%%%%%%%%%%%%%%%%%%%%%%%%%%%%%%%%%%%%%%%%%%%%%%%%%%%%%%%%%%%%%%%%
\subsection{Position Module}

Module \verb|Position| is an abstraction for game rules, like Dots'n'Boxes or Diskonnect.

It defines a new type called \verb|Position| with a certain number of operations 
that each game must implement. 

The operations are:

\begin{itemize} \setlength\itemsep{0.1em}
  \item \verb|points| what is the current scoring
  \item \verb|boards| what is the current board
  \item \verb|moves| what are the legal next moves?
  \item \verb|toText| converts position into a textual description
  \item \verb|fromData| given a number and a textual description, creates a position
\end{itemize}

In this context, a Position value is just the current board position and the current scoring.

To define a new game, the user must implement the previous five functions. However, only function
\verb!moves! has some complexity. The others are quite straightforward. To make a new game, please check the games
already implemented, copy its code and adapt them to your purposes.

These functions only work given concrete game rules. The next egs use Kobber so the user 
must import module \verb!Kobber! which implements Kobber gamerules.

\subsubsection{Preparing a file with a game position}

A practical way to input game positions is to write it on a text file (place it in the same folder
as the modules) and import it directly.

Say we have written file \verb|"kobber1.txt"| like this:

\begin{verbatim}
0
l..
rl.
l.r
\end{verbatim}

The first line is a number with the current score.

The next lines are used to write the board. The representation might vary depending on the 
specific characteristics of the game.

Please use char \verb!l! to represent left pieces, and \verb!r! to represent right 
pieces. The dot is used to represent an empty board cell. Char \verb!'x'! represents a wall.

\subsubsection{evaluating a position in a file}

To read a file with a game position use function \verb!evalG! where \verb!G! is the specific game. In our case, since
we are using Kobber, the function is named \verb!evalKobber!.

\begin{sgcode}
\bash{evalKobber "kobber1.txt"}
\out{-- Read: "0\\nl..\\nrl.\\nl.r"} \\
\out{-- Board} \\
\out{0} \\
\out{l..} \\
\out{rl.} \\
\out{l.r} \\
\out{-- Position Value:} \\
\out{<1|-1>} 
\end{sgcode}

This presents the game value but it does not get assigned to a variable. 

To do that use \verb|fromRaw| and \verb|toGame| with the description that function eval produces in its first line:

\begin{sgcode}
\bash{let pos = fromRaw "1\\nlr.\\nrl.\\nl.r"::Kobber}
\bash{pos}
\out{P {pts = 1.0, board = ["lr.","rl.","l.r"]}} \\
\bash{let g1 = toGame pos}
\bash{g1}
\out{<<3|<2|<1|-1>>>|<1|-1>,<<2|0>|-1,<<2|0>|-1>>>}
\end{sgcode}

\subsubsection{presenting a position}

Function \verb!present! shows a position in a textual format:

\begin{sgcode}
\bash{let pos = fromRaw "1\\nlr.\\nrl.\\nl.r"::Kobber}
\bash{present pos}
\out{lr.} \\
\out{rl.} \\
\out{l.r} \\
\out{1 points} 
\end{sgcode}

Function \verb!presents! does the same but for a list of positions:

\begin{sgcode}
\bash{let pos1 = fromData 0 ["lr","rl",".."]::Kobber}
\bash{let pos2 = fromData 2 ["rr",".."]::Kobber}
\bash{presents \$ [pos1,pos2]}
\out{lr} \\
\out{rl} \\
\out{..} \\
\out{0 points} \\
\out{rr} \\
\out{..} \\
\out{2 points} 
\end{sgcode}

\subsubsection{getting the next possible positions}

Function \verb!moves! returns the next moves given a position and a player.

\begin{sgcode}
\bash{let pos = fromData 0 ["lr","rl",".."]::Kobber}
\bash{presents \$ moves pos Left}
\out{.r} \\
\out{ll} \\
\out{..} \\
\out{1 points} \\
\out{.l} \\
\out{rl} \\
\out{..} \\
\out{1 points} \\
\out{.r} \\
\out{.l} \\
\out{l.} \\
\out{0 points} \\
\out{ll} \\
\out{r.} \\
\out{..} \\
\out{1 points} \\
\out{lr} \\
\out{l.} \\
\out{..} \\
\out{1 points} 
\end{sgcode}

%%%%%%%%%%%%%%%%%%%%%%%%%%%%%%%%%%%%%%%%%%%%%%%%%%%%%%%%%%%%%%%%
\subsection{Games}

\subsubsection{Kobber}

Module \verb|Kobber| implements the game Kobber.

Rules:

Pieces move orthogonally. 
A piece can capture by replacement an adjacent enemy stone (earning a point) or 
capture by jumping over (not earning a point).

\subsubsection{Dots'n'Boxes}

Module \verb|Dots| implements the game Dots'n'Boxes.

Rules:

Starting with an empty grid of dots, players take turns, adding a single horizontal 
or vertical line between two unjoined adjacent dots. A player who completes the 
fourth side of a 1 by 1 box earns one point and takes another turn. 

\begin{sgcode}
\bash{evalDots "dots1.txt"}
\out{-- Read: "1\\nx..xxx\\nxx.xxxx\\nxxx.xx"} \\
\out{-- Board} \\
\out{1} \\
\verb§  x . . x x x § \\
\verb§ x x . x x x x§ \\
\verb§  x x x . x x § \\
\out{-- Position Value:} \\
\out{<^0|^2>} \\
\bash{let pos = fromRaw "1\\nx..xxx\\nxx.xxxx\\nxxx.xx"::Dots}
\bash{present pos}
\verb§ +-+ + +-+-+-+§ \\
\verb§ |#|   | |#|#|§ \\
\verb§ +-+-+-+ +-+-+§ \\
\out{1 points}
\end{sgcode}

\subsubsection{Diskonnect}

Module \verb|Diskonnect| implements the game Dots'n'Boxes.

Rules:

Each move must capture stones.

Stones are captured by jumping orthogonally over enemy stones, checkers-like.

Captures can be multiple, but cannot change the initial direction.

When a player cannot move, she takes the final penalty equal to the number
of her own dead stones (ie, a dead stone is a stone that can still be captured)

\begin{sgcode}
\bash{evalDiskonnect "diskonnect.txt"}
\out{-- Read: "0\\n...\\n..r\\nlr.\\n"} \\
\out{-- Board} \\
\out{0} \\
\out{...} \\
\out{..r} \\
\out{lr.} \\
\out{-- Position Value:} \\
\out{<<2|^2>|^2>} \\
\end{sgcode}

\newpage \section{Annex: How to start}

Goto \url{https://www.haskell.org/platform/}, download and install \textit{The Haskell Platform} 
for your Operating System. 

Clone or download the zip of the current version at \url{https://github.com/jpneto/ScoringGames}.

Enter the GHC interpreter (on Windows choose WinGHCi). Select the Load option, goto the folder with the current modules and position files, and load the adequate module. If you choose a game module, modules \verb!Scoring! and \verb!Position! will also be imported.

It's also possible to create a standalone executable program to perform a sequence of fixed steps. Check file \verb!run_disk.hs! as an eg of how to do it.

\end{document}
